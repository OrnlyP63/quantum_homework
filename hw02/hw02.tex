\documentclass[12pt, a4paper]{article}
\usepackage[utf8]{inputenc}
\usepackage{physics}
\usepackage{graphicx}

%code color
\usepackage{listings}
\usepackage{xcolor}

\definecolor{codegreen}{rgb}{0,0.6,0}
\definecolor{codegray}{rgb}{0.5,0.5,0.5}
\definecolor{codepurple}{rgb}{0.58,0,0.82}
\definecolor{backcolour}{rgb}{0.95,0.95,0.92}

\lstdefinestyle{mystyle}{
	backgroundcolor=\color{backcolour},   
	commentstyle=\color{codegreen},
	keywordstyle=\color{magenta},
	numberstyle=\tiny\color{codegray},
	stringstyle=\color{codepurple},
	basicstyle=\ttfamily\footnotesize,
	breakatwhitespace=false,         
	breaklines=true,                 
	captionpos=b,                    
	keepspaces=true,                 
	numbers=left,                    
	numbersep=5pt,                  
	showspaces=false,                
	showstringspaces=false,
	showtabs=false,                  
	tabsize=2
}

\lstset{style=mystyle}

\begin{document}
	\begin{center}
		Test 2  Entanglement
	\end{center}
	
	Mr. Phiphat Chomchit
	\begin{enumerate}
		\item Suppose you have 5 qubits in the state  $\ket{\Psi} = \frac{1}{\sqrt{2}}\ket{01010} + \ket{10101} .$ Is state $\ket{\Psi}$ entangled? Why?\\
		
		Ans:
		
		\item Write a program to generate $\ket{\Psi}$, run it 1024 times, and come up with an example how to use it in quantum communication.
		\begin{lstlisting}[language=Python, caption=IBM Q code]
			OPENQASM 2.0;
			include "qelib1.inc";
			
			qreg q[5];
			creg c[5];
			
			h q[0];
			x q[1];
			x q[3];
			cx q[0],q[1];
			cx q[0],q[2];
			cx q[0],q[3];
			cx q[0],q[4];
			measure q[0] -> c[0];
			measure q[1] -> c[1];
			measure q[2] -> c[2];
			measure q[3] -> c[3];
			measure q[4] -> c[4];
		\end{lstlisting}
	
			\begin{figure}
				 \centering
				\includegraphics[scale=0.3]{circuit-krik226j.png}
				\caption{quantum circuit}
				\includegraphics[scale=0.2]{bar-chart.png}
				\caption{result}
			\end{figure}

		
	\end{enumerate}
\end{document}